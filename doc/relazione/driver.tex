\section{Il driver}

Il driver è stato sviluppato secondo le specifiche \emph{JDBC 4.0} (JSR 221) del Novembre 2006; secondo queste specifiche il driver è di tipo 4, ossia interamente scritto in java con accesso diretto alle risorse che compongono il database. Per aderire a queste specifiche è necessario che vengano implementate le seguenti interfacce : 
\begin{itemize}
\item[-] java.sql.Driver
\item[-] java.sql.DatabaseMetaData
\item[-] java.sql.ParameterMetaData
\item[-] java.sql.ResultSetMetaData
\item[-] java.sql.Wrapper
\item[-] java.sql.DataSource
\end{itemize}
e supportare nei ResultSet la capacita di concorrenza in lettura. 

%Java Specification 

%-- immagine riassuntiva di come è strutturato un driver JDBC 

%-- classi introdotte 

 %--- Come interprete/risolutore prolog è stato scelto il tuProlog nella sua ultima versione disponibile. ---
 
% -- se manca il metabase, ne creo cmq. uno io all'avvio e lo tengo in memoria ---- 

\subsection{Struttura}

\subsection{Operazion}	% CRUD

\subsection{Transazioni}	% ACID