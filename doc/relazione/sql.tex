\section{Structured Query Language}

Lo Structured Query Language o SQL in forma abbreviata, è un linguaggio standard utilizzato per interrogare i database relazionali; allo stato attuale i moderni database implementano solo la parte \emph{Entry Level} di tale standard. 

\subsection{Il parser}
Per poter interpretare le richieste sql dell'utente si è reso necessario la creazione di un parser; in particolare dovrà essere in grado di poter riconoscere un sottoinsieme dello standard SQL/92. La sua realizzazione è avvenuta usando un tool chiamato javacc della Sun che mi ha permesso di specificare solo la grammatica da riconoscere e di lasciare allo strumento la generazione del codice relativa all'effettiva analisi del testo. La grammatica utilizzata per generare il parser comprende :
\begin{itemize}
\item {\bf Operatori del linguaggio}, ossia le parole significative, che nel nostro caso possono essere di due tipi : 
\begin{itemize}
\item[-] Comandi
\item[-] Operatori SQL
\end{itemize}
\item {\bf Tipi di dato}
\end{itemize}

\subsection{Comandi}
I Comandi possibili sono suddivisi in due categorie Data Definition Language o DDL e Data Manipulation Language o DML. Quelli che ho implementato sono:
\begin{itemize}
\item DDL :
\begin{itemize}
\item[-] Create "Database"
\item[-] Drop "Database"
\end{itemize}
\item DML :
\begin{itemize}
\item[-] Create Table
\item[-] Select
\item[-] Update
\item[-] Insert 
\item[-] Delete
\item[-] Truncate Table 
\item[-] Drop Table
\end{itemize}
\end{itemize}

\subsection{Operatori}
Gli operatori, messi a disposizione dal SQL/92 si dividono in quattro categorie:
\begin{enumerate}
\item Operatori di confronto
\item Operatori aritmetici
\item Operatori condizionali
\item Operatori logici
\end{enumerate}
Di questi sono stati presi in considerazione solo quelli di confronto, che sono solitamente i più utili. 