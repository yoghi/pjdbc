\section{Il database}

Una definizione di database può essere {\bf il database rappresenta un archivio di dati strutturati}; la struttura più diffusa in questo momento è quella chiamata \emph{relazionale}, che si compone di tabelle e di relazioni tra di esse.

\subsection{Database Prolog}
Il database avrà una struttura relazionale e sarà contenuto in una teoria prolog; le relazioni saranno espresse mediante l'uso di opportuni predicati. La tabella all'interno di questo database sarà rappresenta da un insieme di termini \emph{compound} prolog che rappresenteranno le righe delle tabelle; le righe avranno come sintassi:  
\begin{lstlisting}[language=Prolog]
 nomeTabella(col1,col2,......,colN).
\end{lstlisting}
I termini per appartenere ad una tabella dovranno avere come nome il nome della tabella. Esempio:
\begin{lstlisting}[language=Prolog]
 ta(col1,col2,col3).
 ta(col11,col21,col31).
\end{lstlisting}
questi due termini rappresentano due righe di una tabella;
\begin{lstlisting}[language=Prolog]
 ta(col1,col2,col3).
 tb(col1,col2,col3).
\end{lstlisting}
questi due termini rappresentano due righe di due tabelle rispettivamente ta e tb;
\begin{lstlisting}[language=Prolog]
 ta(col1,col2,col3).
 ta(col1,col2).
\end{lstlisting}
questi due termini sono ambigui, non si sa cosa si voglia rappresentare; potrebbe essere che la col3 è null e quindi è stata omessa? la seconda tabella è una tabella diversa dalla prima? per semplificare ritengo che nel caso si presenti una situazione del genere venga generata un'eccezione e si lasci all'utente decidere come disambinguare la situazione. Per esprimere che una colonna è non settata basta usare il valore \emph{NULL} come valore, esempio : 

\begin{lstlisting}[language=Prolog]
 ta(col1,col2,col3).
 ta(col1,NULL,col2).
\end{lstlisting} 

\subsection{Il metabase}
Per avere la completezza di informazioni tipiche di un moderno database relazionale è necessario avere delle ulteriori informazioni su come è strutturato il database ed in particolare sulla struttura delle tabelle; queste informazioni solitamente compongono quello che è chiamato \emph{metabase} o \emph{metadatabase}. Un possibile insieme di informazioni utili possono essere : 
\begin{enumerate}
\item il nome da associare ad una colonna
\item il tipo di dato che contiene una colonna
\item la descrizione del contenuto di una colonna 
\end{enumerate}
questi sono solo alcuni possibili elementi. Nel nostro caso il \emph{metabase} può essere descritto attraverso l'uso di una tabella di sistema dal nome {\bf metabase}; la sintassi sarà : 
\begin{lstlisting}[language=Prolog,showstringspaces=false]
 metabase(
	"nome tabella",
	"posizione della colonna tra gli argomenti",
	"nome della colonna",
	"tipo di dato contenuto",
 ).
\end{lstlisting}
Questa particolare tabella non è pero necessaria, il database sarà comuque consistente, ma la sua presenza facilita l'interazione dell'utente nello scrivere query sql. Un esempio :
\begin{lstlisting}[language=Prolog,showstringspaces=false]
 metabase("ta",1,"colonna1","string").
\end{lstlisting}
Permetterà all'utente di usare il nome della colonna nelle clausole sql.

\subsection{Tipi di dato}

\subsubsection{Dati primitivi e Strutture Dati}

Il database essendo scritto in prolog avrà come tipi di dati primitivi gli stessi supportati dal prolog; ossia \emph {int,double,float,string} con l'aggiunta dei \emph{boolean}: true e false. 
Come strutture dati ho preso in considerazione gli Array che posso tranquillamente mappare come \emph{List} in Prolog.

