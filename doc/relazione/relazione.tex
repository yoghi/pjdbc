
\documentclass[a4paper,titlepage,12pt]{article}	% documento A4, carattere 12pt , dopo il titolo si passa ad una pagina nuova
%\documentclass[a4paper,12pt,twoside]{article}	% documento A4, ottimizzato fronte-retro

\usepackage{graphicx} 				%inserisco la grafica nel documento
\usepackage{booktabs} 				% for much better looking tables
\usepackage{array} 					% for better arrays (eg matrices) in maths
\usepackage{paralist} 				% very flexible & customisable lists (eg. enumerate/itemize, etc.)
\usepackage[italian]{babel} 			%[riferimento all'utilizzo di LaTeX in italiano].
\usepackage[T1]{fontenc} 			% uso caratteri con lo stesso aspetto su ogni pc usabile
\usepackage[utf8x]{inputenc} 			% codifica : utf8 
\usepackage{lipsum} 				% inserisco carateri a caso per poter formattare la paginazione
\usepackage{listings}				% inserisco codice da visualizzare come testo

%le mie definzione
\makeatletter
\def\maketitle{%
  \null
  \thispagestyle{empty}%
  \vfill
  \begin{center}\leavevmode
    \normalfont
    {\huge \raggedright \@title \par}%
    \hrulefill \par
    di {\LARGE \raggedleft \@author \par }%
    {\raggedleft Matricola 0000273180 \par}%
    \vskip 3cm
    Corso : Linguaggi e Modelli Computazionali 2006/07
    \vskip 1cm
    last update : { \@date\par}%
  \end{center}%
  \vfill
  \null
  \cleardoublepage
  }
\makeatother


% Titolo del corso, autore, data
\title{Realizzazione di un driver JDBC per l'accesso ad un database scritto in prolog}
\author{Stefano Tamagnini} 
\date{\today}

%enfatizzo i metodi square,root
\lstset{ %
language=Java 		%linguaggio
%,frame=single			%frame
}


% metto i rif. di pagina in alto invece che in baso nelle pagine
%\pagestyle{headings}

\begin{document}

\maketitle

\tableofcontents

\newpage

\section{Introduzione}

Come progetto per l'esame di Linguaggi e Modelli Computazionali, ho realizzato un driver JDBC che permetta di poter eseguire query SQL su un database scritto in linguaggio Prolog. Della sintassi SQL che è molto vasta ho realizzato le funzionalità di base che permettano : Creation, Read, Update, Delete (CRUD), lasciando l'implementazione completa secondo lo standard SQL ad un secondo momento.
\par 
Questo driver, ho deciso di chiamarlo {\bf PJDBC}, dove la lettera P sta appunto per Prolog ossia il linguaggio in cui è espresso il database da interrogare; le parti coinvolte in questo progetto sono : 
\begin{enumerate}
\item il database
\item il linguaggio con cui interrogarlo (sql)
\item il driver jdbc che permetta l'interazione del database con le richieste dell'utente
\end{enumerate}

\section{Il database}

Una definizione di database può essere {\bf il database rappresenta un archivio di dati strutturati}; la struttura più diffusa in questo momento è quella chiamata \emph{relazionale}, che si compone di tabelle e di relazioni tra di esse.

\subsection{Database Prolog}
Il database sarà una teoria prolog, dove le tabelle verranno espresse come una lista di fatti (\emph{facts}) dove ogni atomo della lista rappresenterà una riga di tale tabella. Il nome della tabella sarà il predicato mentre gli argomenti saranno i valori delle colonne. Per rappresentare una tabella di nome nomeTabella con N colonne (col1...colN) potrò usare la seguente sintassi:
\begin{lstlisting}[language=Prolog]
 nomeTabella(col1,col2,......,colN).
\end{lstlisting}
Un caso particolare può essere: 
\begin{lstlisting}[language=Prolog]
 tabella(col1,col2,col3).
 tabella(col1,col2).
\end{lstlisting}
questi due termini sono ambigui, non si sa cosa si voglia rappresentare; potrebbe essere che la col3 è null e quindi è stata omessa? oppure la seconda tabella è una tabella diversa dalla prima? per semplificare ho retenuto che nel caso si presenti una situazione del genere venga generata un'eccezione e si lasci all'utente decidere come disambinguare la situazione. Un modo per risolvere potrebbe essere quello usare il termine \emph{null} come valore, esempio : 

\begin{lstlisting}[language=Prolog]
 ta(col1,col2,col3).
 ta(col1,null,col2).
\end{lstlisting} 

Il termine \emph{null} ovviamente dovrà essere minuscolo altrimenti dal prolog verrebbe considerato una variabile. Il database così costruito ha come tipi di dati primitivi gli stessi supportati dal prolog; come strutture dati ha anche il supporto agli Array in quanto sono facilmente rappresentabili in prolog come delle liste. 

\subsection{Il metabase}
Per avere la completezza di informazioni tipiche di un moderno database relazionale è necessario avere delle ulteriori informazioni su come è strutturato il database ed in particolare sulla struttura delle tabelle; queste informazioni solitamente compongono quello che è chiamato \emph{metabase} o \emph{metadatabase}. Un possibile insieme di informazioni utili possono essere : 
\begin{enumerate}
\item il nome da associare ad una colonna
\item il tipo di dato che contiene una colonna
\item la descrizione del contenuto di una colonna 
\end{enumerate}
questi sono solo alcuni possibili elementi. Nel nostro caso il \emph{metabase} può essere descritto attraverso l'uso di una tabella di sistema dal nome {\bf mtable}; la sintassi sarà : 
\begin{lstlisting}[language=Prolog,showstringspaces=false]
 mtable(
	"nome tabella",
	"posizione della colonna tra gli argomenti",
	"nome della colonna",
	"tipo di dato contenuto",
 ).
\end{lstlisting}
Questa particolare tabella non è pero necessaria, il database sarà comuque consistente, ma la sua presenza facilita l'interazione dell'utente nello scrivere query sql. Un esempio :
\begin{lstlisting}[language=Prolog,showstringspaces=false]
 mtable("ta",1,"colonna1","string").
\end{lstlisting}
Permetterà all'utente di usare il nome della colonna nelle clausole sql.

\subsection{Store Procedure e Function}
E' possibile pensare a far supportare al database prolog anche le \emph{store procedure} e le \emph{function}  in quanto in una teoria prolog oltre ai fatti si possono inserire anche delle regole (\emph{rules}). Il supporto a tale funzionalità è stato rimandato a estensioni future. 


\section{Structured Query Language}

Lo Structured Query Language o SQL in forma abbreviata, è un linguaggio standard utilizzato per interrogare i database relazionali; allo stato attuale i moderni database implementano solo la parte \emph{Entry Level} di tale standard. 

\subsection{Il parser}
Per poter interpretare le richieste sql dell'utente si è reso necessario la creazione di un parser; in particolare dovrà essere in grado di poter riconoscere un sottoinsieme dello standard SQL/92. La sua realizzazione è avvenuta attraverso lo strumento javacc che ha permesso di specificare solo la grammatica da riconoscere e di lasciare allo strumento la generazione del codice relativa all'effettiva analisi del testo. La grammatica utilizzata per generare il parser comprende :
\begin{itemize}
\item {\bf Operatori del linguaggio}, ossia le parole significative, che nel nostro caso possono essere di due tipi : 
\begin{itemize}
\item[-] Comandi
\item[-] Operatori SQL
\end{itemize}
\item {\bf Tipi di dato}
\end{itemize}

\subsection{Comandi}
I Comandi possibili sono suddivisi in due categorie Data Definition Language o DDL e Data Manipulation Language o DML. Quelli che ho ritenuto necessari implementare sono:
\begin{itemize}
\item DDL :
\begin{itemize}
\item[-] Create "Database"
\item[-] Drop "Database"
\end{itemize}
\item DML :
\begin{itemize}
\item[-] Select
\item[-] Update
\item[-] Insert 
\item[-] Delete
\item[-] Truncate Table
\item[-] Drop Table
\end{itemize}
\end{itemize}
Come si nota non ho inserito \emph{Create Table} in quanto già la \emph{Insert} permette di creare una tabella all'interno del database prolog; ovviamente in questo caso mancano i metadati ma possono essere inseriti manualmente con delle \emph{Insert} direttamente con riferimento alla tabella metabase. 

\subsection{Operatori}
Gli operatori, messi a disposizione dal SQL/92 si dividono in quattro categorie:
\begin{enumerate}
\item Operatori di confronto
\item Operatori aritmetici
\item Operatori condizionali
\item Operatori logici
\end{enumerate}
Di questi sono stati presi in considerazione quelli aritmetici e quelli di confronto 

\section{Il driver}

Il driver è stato sviluppato secondo le specifiche \emph{JDBC 4.0} (JSR 221) del Novembre 2006; secondo queste specifiche il driver è di tipo 4, ossia interamente scritto in java con accesso diretto alle risorse che compongono il database. Per aderire a queste specifiche è necessario che vengano implementate le seguenti interfacce : 
\begin{itemize}
\item[-] java.sql.Driver
\item[-] java.sql.DataSource
\item[-] java.sql.DatabaseMetaData
\item[-] java.sql.ParameterMetaData
\item[-] java.sql.ResultSetMetaData
\item[-] java.sql.Wrapper
\end{itemize}
e supportare nei ResultSet la capacita di concorrenza in lettura. 

%Java Specification 

%-- immagine riassuntiva di come è strutturato un driver JDBC 

%-- classi introdotte 

 %--- Come interprete/risolutore prolog è stato scelto il tuProlog nella sua ultima versione disponibile. ---
 
% -- se manca il metabase, ne creo cmq. uno io all'avvio e lo tengo in memoria ---- 

\subsection{Struttura}

Catalog è la cartella in cui si opera
Schema è il nome del file .db su cui si opera
Table è il funtore che si usa all'interno della teoria prolog

Nella cartella ci sono anche dei file .log(??) che servono per tenere traccia delle transizioni

\subsection{Operazion}	% CRUD

\subsection{Transazioni}	% ACID
	
\end{document}